\documentclass[12pt, letterpaper, twoside]{article}
\usepackage[utf8]{inputenc}

\title{Math}
\begin{document}
	
	\maketitle
\section{Problem 1}
$a$ in all forms purely changes the shape and direction of the parabola. $c$ in the expanded form is the y-intercept. $d$ in the vertex form is the $x$ coordinate of the vertex. $h$ in vertex form is the $y$ coordinate of the vertex. $x_1$ and $x_2$ in the factored form represent the two x-intercepts. Increasing $b$ moves the vertex of the parabola left while maintaining the same shape and y-intercept. Decreasing $b$ moves the vertex of the parabola right while maintaining the same shape and y-intercept.
\section{Problem 2}
A quadratic does not have a factored form if its $a$ is positive and its y-intercept is greater then 0 or if its $a$ is negative and its y-intercept is less then $0$.
\section{Problem 3}
$x_1$ and $x_2$ are solutions to the equation $0 = ax^2 + bx + c$, where $x$ is $x_1$ and $x_2$ respectively, the equation $0 = a(x-d)^2 - h$, where $x$ is $x_1$ and $x_2$ respectively and the equation $0 = a(x-x_1)(x-x_2)$.
\section{Problem 4}
\subsection{Expanded to Vertex}
$a$ is the same in both forms. $d$ in vertex is $\frac{b}{2a}$. h is $c - \frac{b^2}{4a^2}$.
\subsection{Expanded to Factored}
$a$ is the same in both forms. $x_1$ and $x_2$ are the two solutions to $0 = ax^2+bx+c$.
\subsection{Vertex to Expanded}
$a$ is the same in both forms. Evaluate $(x+d)^2$, then add $d^2$ to c and subtract remove the $d^2$ term. 
\subsection{Vertex to Factored}
$a$ is the same in both forms. $x_1$ and $x_2$ are the two solutions to $0 = a(x+d)^2 + h$
\subsection{Factored to Expanded}
$a$ is the same in both forms. Evaluate the $(x-x_1)(x-x_2)$ to find the expanded form.
\subsection{Factored to Vertex}
Convert to expanded form then to vertex form.
\section{Problem 4}
Let $(x_0,y_0)$ be a solution to the equation $y = f(x)$. If you change the $x$ to $x-s$, increasing $x_0$ by $s$ to account for the decrease by $s$ would also be a solution. Therefore $x_0 + s, y_0$ is a solution to $y = f(x-s)$. If you change $y$ to $y-t$, if you added $t$ to $x_0$ it would cancel out the $t$, therefore $(x_0, y_0 + t)$ is a solution. If we combine the two we find that $(x_0 + s, y_0 + t)$ is always a solution to $y-t =  f(x-s)$ if $(x_0,y_0)$ is a solution to $y = f(x)$. This means that if you increase $s$, every point moves right, and if you increase $t$ every point moves up.
\end{document}