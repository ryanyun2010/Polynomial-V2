\documentclass[12pt, letterpaper, twoside]{article}
\usepackage[utf8]{inputenc}

\title{Math}
\begin{document}
	
	\maketitle
\section{Problem 1}
A polynomial is an expression with at least one variable. All powers in a polynomial must be non-negative integers. The degree of a polynomial is the degree of the term with the largest degree. The degree of a term is the exponents on the variables of the term added up.\par
For example the degree of the term $3x^2$, assuming $x$ is a variable, would be $2$ as $x$ is the only variable and its exponent is $2$. Similiarly the degree of $3x^2 + 5x +2$ is $2$ as $3x^2$ is the term with the largest degree $2$. \par
Another example is the term $x^2y$, assuming both $x$ and $y$ are variables, the degree of the term is $3$ as the exponent of $x$ is $2$ and the exponent of $y$ is $1$. $2 + 1$ is $3$. Simillarily the degree of $2x^2y + x^2 + 2xy + 5$ is $3$. As the degree of the largest term, $2x^2y$ is $3$. 
\section{Problem 2}
$3^(3.5)x^2y^3z^6 + z^2$ is not a polynomial as there is a non-integer power $3.5$
\section{Problem 3}
If $a$ and $b$ are both parameters then there is $1$ variable and the degree is $2$. If $a$ is a variable and $b$ is a parameter then there are $2$ variables and the degree is $3$. If $b$ is a variable and $a$ is a parameter then there are $2$ variables and the degree is $4$. If $a$ and $b$ are both variables then there are $3$ variables and the degree is $4$. In all cases it is a polynomial
\section{Problem 4}
The degree is $2$ as $x^2$ is the term with the heighest degree.
\section{Problem 5}
If $a$ is a parameter then it cannot change within the problem, if $a$ is a variable it can change. The same applies to $x$. $x$ is generally a variable and $a$ is generally a parameter. Meaning the difference between the two is that $a$ cannot change within the problem but $x$ can.	
\end{document}