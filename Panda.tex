\documentclass[12pt, letterpaper, twoside]{article}
\usepackage[utf8]{inputenc}

\title{Math}
\begin{document}
	
	\maketitle
\section{Problem 1}
A polynomial is an expression with at least one variable. All powers in a polynomial must be non-negative integers. The degree of a polynomial is the degree of the term with the largest degree. The degree of a term is the exponents on the variables of the term added up.\par
For example the degree of the term $3x^2$, assuming $x$ is a variable, would be $2$ as $x$ is the only variable and its exponent is $2$. Similiarly the degree of $3x^2 + 5x +2$ is $2$ as $3x^2$ is the term with the largest degree $2$. \par
Another example is the term $x^2y$, assuming both $x$ and $y$ are variables, the degree of the term is $3$ as the exponent of $x$ is $2$ and the exponent of $y$ is $1$. $2 + 1$ is $3$. Simillarily the degree of $2x^2y + x^2 + 2xy + 5$ is $3$. As the degree of the largest term, $2x^2y$ is $3$. 
\section{Problem 2}
The degree of $ax^2 + x$ is $2$ as the term with the largest degree is $ax^2 + x$.
\section{Problem 3}
$a$ only affects the term $ax^2$, whereas $x$ effects both terms. $a$ is also usually a parameter, whereas $x$ is usually a variable. The speed at which $x$ increases/decreases compared to $y$ decreases as $a$ gets larger.
\section{Problem 4}
$(0,0)$ is special as its the vertex as well as the x and y intercepts
\section{Problem 5}
When $a$ is positive, as $a$ gets larger, any given point aside from $(0,0)$ gets closer to the line $x = 0$. When $a$ is negative, as $a$ gets smaller, any given point aside from $(0,0)$ gets closer to the line $x = 0$. When $a$ is $0$ it is just a straight line $y=0$
\end{document}